\clearpage
\noindent\textit{Title:}\\
\textbf{Dual Circularly Polarized Waveguide Antenna}\\[0.25cm]
\textit{Author:} Martin Šimák\\[0.25cm]
\textit{Study programme:} Electronics and Communications\\[0.25cm]
\textit{Supervisor:} doc. Ing. Pavel Hazdra, Ph.D., Department of Electromagnetic Field FEE\\[0.25cm]
\textit{Abstract:}\\
This thesis details the design, simulation, fabrication, and measurement of a novel dual circularly polarized antenna operating in the $\frequencyrange$ band. The system comprises a square waveguide polarizer with chamfered corners, a dual-coaxial feed, and a conical horn antenna. The polarizer generates right-hand and left-hand circular polarization by selectively exciting one of the two fundamental modes of the square waveguide. The dual coaxial feed provides the necessary excitation. The conical horn, designed using Antenna Magus and adapted to the polarizer, achieves a target gain of $\qty{15}{dBi}$. CST Studio Suite was used for electromagnetic simulations, while Python with SciPy enabled dynamic optimization for enforcing geometric constraints. The fabricated antenna's measured performance closely aligns with simulations, demonstrating an axial ratio below $\qty{5}{dB}$ across the band and a measured gain of approximately $\qty{18}{dBi}$ at the centre frequency. This work contributes to a compact, manufacturable, dual circularly polarized antenna design with potential applications in satellite communications, radar, and other wireless systems.\\[0.25cm]
\textit{Keywords:} circular polarization, waveguide polarizer, dual-feed, conical horn antenna, hexagonal waveguide, eigenmode analysis, electromagnetic simulation\\[0.5cm]
\begin{otherlanguage}{czech}
    \noindent\textit{Název práce:}\\
    \textbf{Duálně kruhově polarizovaná vlnovodová anténa}\\[0.25cm]
    \textit{Autor:} Martin Šimák\\[0.25cm]
    \textit{Studijní program:} Elektronika a komunikace\\[0.25cm]
    \textit{Vedoucí:} doc. Ing. Pavel Hazdra, Ph.D., Katedra elektromagnetického pole FEL\\[0.25cm]
    \textit{Abstrakt:}\\
    Tato diplomová práce popisuje návrh, simulaci, výrobu a měření duální kruhově polarizované antény pracující v pásmu $\frekvencnipasmo$. Systém se skládá ze čtvercového vlnovodného polarizátoru se zkosenými rohy, dvojitého koaxiálního napájení a kuželové antény. Polarizátor generuje pravotočivou a levotočivou kruhovou polarizaci selektivně buzené jedním ze dvou základních módů čtvercového vlnovodu. Potřebné buzení zajišťuje duální koaxiální napájení. Kuželová anténa navržená pomocí programu Antenna Magus a přizpůsobená polarizátoru dosahuje cílového zisku $\qty{15}{dBi}$. Pro elektromagnetické simulace byla použit software CST Studio Suite, zatímco Python s použitím knihovny SciPy umožnil dynamickou optimalizaci pro vynucení geometrických omezení. Naměřený výkon zhotovené antény se přesně shoduje se simulacemi a vykazuje osový poměr pod $\qty{5}{dBi}$ v celém pásmu a naměřený zisk přibližně $\qty{18}{dBi}$ na střední frekvenci. Tato práce přináší kompaktní, vyrobitelnou konstrukci duální kruhově polarizované antény s potenciálním využitím v satelitní komunikaci, radarové technice a dalších bezdrátových systémech.\\[0.25cm]
    \textit{Klíčová slova:} kruhová polarizace, vlnovodný polarizátor, duální napájení, kuželová anténa, hexagonální vlnovod, analýza vlastních módů, elektromagnetická simulace\\[0.5cm]
\end{otherlanguage}
\clearpage